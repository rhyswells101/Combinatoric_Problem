\documentclass[a4paper,12pt]{article}
\newcounter{example}[]
\newenvironment{example}[1][]{\refstepcounter{example}\par\medskip
   \noindent \textbf{Example~\theexample. #1} \rmfamily}{\medskip}
   %%%%%%%%%%%%%%%%%%%%%%%%%%%%%%%%

%%%%%%%%%%%%%%%%%%%%%%%%%%%%%%%%%%
\usepackage[utf8]{inputenc}
\usepackage[english]{babel}
\usepackage{tikz-cd}
\usepackage{amsmath,amsfonts,amssymb,amsthm}
\usepackage{mathtools}
 \usepackage{float}
\usepackage{amsthm}
\usepackage{cite}
\usepackage{datetime} % British format dates
\usepackage[cm]{fullpage}
\usepackage{url}
\usepackage{hyperref}
\usepackage{stackrel,amssymb,amsmath}
\usepackage[nottoc]{tocbibind}
\usepackage{pgfplots}
\usepackage{rotating}
\usepackage[autostyle]{csquotes}
\usepackage{natbib}
\usepackage{graphicx}
\usepackage{natbib}
\usepackage{graphicx}

\newtheorem{problem}{Problem}
\newtheorem{attempt}{Attempt}


\newtheorem{theorem}{Theorem}[section]
\newtheorem{corollary}{Corollary}[theorem]
\newtheorem{lemma}[theorem]{Lemma}
\newtheorem{proposition}[theorem]{Proposition}
\theoremstyle{definition}
\newtheorem{definition}{Definition}[section]
\theoremstyle{indented}
\newtheorem*{remark}{Remark}
\newenvironment{titlemize}[1]{%
  \paragraph{#1}
  \begin{itemize}}
  {\end{itemize}}
  
  \usepackage[T1]{fontenc}
\usepackage{imakeidx}
\makeindex[columns=3, title=Alphabetical Index, intoc]
  
  
  %%%%%%%%%%5
\newcommand{\rightarrowdbl}{\rightarrow\mathrel{\mkern-14mu}\rightarrow}

\newcommand{\xrightarrowdbl}[2][]{%
  \xrightarrow[#1]{#2}\mathrel{\mkern-14mu}\rightarrow
}
%%%%%%%%%%%%%%5

\title{Group action of hyperplane family}
\author{Rhys Wells}
\date{\today}

\begin{document}

\maketitle

\begin{titlemize}{Objectives}

\item Problem 1: Replace coordinates of $\underline{x}$ in $W_D(l,S,k)$ by new coordinates $\lambda(\phi,\underlinee{x})$ and check if it satisfies the equation determining $W_D(l^{'},S^{'},k^{'})$.

 
\end{titlemize}

\hline
The are the notes from 30/7/20. The material in the rest of the doc needs to be changed to fit these notes.

\begin{itemize}
    \item Note $d \ne \sum d_i$, but the total degree of the line bundle is $d= \omega (2g-2) + \sum d_i$ with twisting.

\item We are \textbf{changing $\omega$ to $s$} (in the literature usually called $l$ or $k$).

\item The action of $\lambda$ does not change the coordinates this fixes $S$ (as it reflects and translates diherdral group) changes only the $k$ of $W_D(l,k,S)$. The action fixes $d$.

\item Seperate the cases $g=1$ and $g \ge 2$. In $g=1$ have $x_S=k$ what happens when apply $\lambda$? Check that For $t=0$ only $k$ changes. For $g\ge 2$ have 

$$ \lambda((\underline{d},s,t), \underline{x})= \underline{d} +s(2g-2)  \begin{bmatrix}
           1 \\
           \vdots \\
           1
         \end{bmatrix}
 +(-1)^t \underline{x}$$


we have 

$$W_D(l,S,k):= \left\{ \underline{x} \mid x_S + \frac{ l(d+1-g - x_{[n] } ) } {2g-2} = k \right\}. $$

have $x_{S} \mapsto s(2g-2)|S| + -1)^r x_{S} + d_S$

substituting new corrdinates we get;

$$s(2g-2)|S| + (-1)^t x_{S} + d_{S} +\frac{ l(d+1-g -(s(2g-2) n +(-1)^t x_{[n]} + d_{[n]} ) } {2g-2} = k$$ 

the terms $s(2g-2)|S|$ and  $d_{S}$ change the value of $k$ and the term $s(2g-2) n$ in the fraction changes $k$ and $l$. The $d$ in $W_D(l,S,k)$ definition is unrealted to $\underline{d}$ in $\lambda((\underline{d},s,t), \underline{x})$. \textbf{Call $\underline{d}$ now $\underline{e}$ with $e_i$}

Note when $d=0$ then get $\frac{1-g}{2g-2} = 1/2$. 

\item We take $d=0$ with the benifit that $\Tilde{PR}^{0}$ is explicitley written as in pdf. 

when apply the $\lambda$ get total degree change by $k\textbf{n}(2g-2) +\sum d_i =0$ 


\end{itemize}





\hline 


\begin{definition}\label{planefam}  For $0\le l \le 2g-2 + \delta_{1,g}$, $S\subseteq [n] \setminus \{\delta_{1,g}\}$ and $k\in \mathbb{Z}$ excluding $l=0$ and $S=\emptyset$. We have the following family of hyperplanes,

$$W_D(l,S,k):= \left\{ \underline{x} \mid x_S + \frac{ l(d+1-g - x_{[n]\setminus{\delta_{1,g}} } ) } {2g-2+\delta_{1,g}} = k \right\}. $$
\end{definition}

Consider the case $g \ge 2$ and $d=0$. Define $P_{g,n}$ as $$P_{g,n}= \{(d_1,\dots,d_n,\omega ) \in \mathbb{Z}^{n+1} \mid \omega (2g-2) + \sum d_i=0\}.$$ I have replaced $k$ is this equation (written in pdf) with $\omega$ to avoid confusion with $W_D(l,k,S)$. 


\begin{titlemize}{Questions}

\item  
Note (Possible confusion here), under the conditions $g \ge 2$ and $d=0$ one has $\omega =0$. On page 35 remark 6.21 \cite{kass2019stability}, one defines $\Tilde{P}_{g,n}$ first, then we ask for $d=0=\sum d_i$. Then for the point $(d_1,\dots,d_n,\omega ) \in P_{g,n}$ one has $\omega (2g-2) + \sum d_i=0$, which implies $\omega =0$ as $g \ge 2$. Is $\omega =0$ in the $g \ge 2$ and $d=0$ case?


\end{titlemize}


\begin{problem}
For $\phi = (\underline{d},\omega,t) \in \Tilde{P}_{g,n}$, show that $\lambda(\phi, W(l,k,S))=W(l^{'} ,k^{'},S^{'})$ for some $l^{'} ,k^{'},S^{'}$.\\

This problem can be simplified by taking $g \ge 2$ and $d=g-1$ (or potentially $d=0$). 

\end{problem}

\begin{attempt}

Consider the case $g=1,n=3$ one has the following, $l=0$ and

$$W_D(0,\{2\}, k) = \left\{ \underline{x} \mid x_2=k\right\} $$

$$W_D(0,\{3\}, k^{'}) = \left\{ \underline{x} \mid x_3=k^{'}\right\} $$


$$W_D(0,\{2,3\}, k^{''}) = \left\{ \underline{x} \mid x_2 + x_3= k^{''}\right\} $$

and $\lambda(\phi, \underline{x}) = (-1)^t + \underline{d}$, with $\underline{d} \in \mathbb{Z}^{n-1}$. Choose a point that belongs to a particular hyperplane family and consider the action of $\lambda$ for arbitary $\phi$ on the point. After this action does the point belong to another $W_D$ with different $l,S,k$? 

\hfill \break

Let $ \underline{x} \in W_D(0,\{2\}, k)$, then $\lambda (\phi, (k,x_3) ) = ( (-1)^{t} k +d_1,  (-1)^{t} x_3 +d_2 ) $. We have $x_2 = (-1)^{t} k +d_1$ so $\lambda (\phi, (k,x_3) ) \in W_D(0,\{2\}, k)$ as $k \in \mathbb{Z}$. It does not seem like the $l$ or $S$ change.
\end{attempt}


\begin{attempt}

Let $g \ge 1$ and let $d$ be arbitrary (not yet $0$ due to the issue in the Question above). Want to show we can show the problem in this case.

\begin{titlemize}{Question}
  \item are $P_{g,n}$ and $\Tilde{P}_{g,n}$ defined in the pdf for $g \ge 2$ and $d=0$ specifically? 
\end{titlemize} 


Let $\underline{x} \in W(l,k,S)$ then $\lambda(\phi, \underline{x}) = (d_1 +\omega(2g-2) + (-1)^{t} x_1, \dots ,d_n +\omega(2g-2) + (-1)^{t} x_n )$. Insert $\lambda(\phi, \underline{x})$ into 

\begin{equation} \label{hypereq}
    x_S + \frac{ l(d+1-g - x_{[n]\setminus{\delta_{1,g}} } ) } {2g-2+\delta_{1,g}} = k.
\end{equation}


Then ask, does the result satisfy the equation determining the hyperplane for $W(l^{'} ,k^{'},S^{'})$? Without loss of generality (forgot the ordering on $S$) take $S=\{1,2,\dots,r\}$ with $|S|=r$. Multiply equation (\ref{hypereq}), by the denominator $2g-2+ \delta_{1,g}$. 


$$ (2g-2+\delta_{1,g})x_S +  l(d+1-g - x_{[n]\setminus{\delta_{1,g}} } ) = (2g-2+\delta_{1,g}) k$$

Recall $x_S= \sum_1^r x_i $ and $d=\sum^{n}_1 d_i$. Replace $x_i$ terms by those after the action of $\lambda$. Then,


$$ (2g-2+\delta_{1,g})(\sum_{i=1}^{r} (d_i+\omega(2g-2)+(-1)^{t} x_i)) +  l(d+1-g - \sum_{i={1+\delta_{1,g}}} ^{n} (d_i+\omega(2g-2)+(-1)^{t} x_i  ) ) = (2g-2+\delta_{1,g}) k$$
 
 \begin{titlemize}{Question}
 \item  Can we write this back into the form of equation (\ref{hypereq})? (attempted rearranging but got confused with use of $\delta__{1,g}$).
 \end{titlemize}

 

% \hline
%issue here with g=1 case and $\delta_{1,g}$ below is not correct.
 %$$ (2g-2+\delta_{1,g})(\sum_{i=1}^{r} d_i +\omega(2g-2)r+(-1)^{t} \sum_{i=1}^{r} x_i) +  l(d+1-g -\sum_{i={1+\delta_{1,g}}} ^{n} d_i - (n-\delta_{1,g})\omega(2g-2)-(-1)^{t} \sum_{i=1+\delta_{1,g}}^n x_i   ) = (2g-2+\delta_{1,g}) k$$
%\begin{equation}
%\begin{align}
 %     (2g-2+\delta_{1,g})(\sum_{i=1}^{r} d_i &+\omega(2g-2)r) +(2g-2+\delta_{1,g})(-1)^{t} \sum_{i=1}^{r} x_i)\\
  %    &+ l(d+1-g -(-1)^{t} \sum_{i=1}^n x_i)+l(-\sum_{i={1+\delta_{1,g}}} ^{n} d_i - n\omega(2g-2) )\\
%      &=(2g-2+\delta_{1,g}) k 
%\end{align}
%\end{equation}
%\begin{equation}
%\begin{align}
 %     (2g-2+\delta_{1,g})(-1)^{t} \sum_{i=1}^{r} x_i)&+ l(d+1-g -(-1)^{t} \sum_{i=1}^n x_i)\\
  %    &=(2g-2+\delta_{1,g}) k - (2g-2+\delta_{1,g})(\sum_{i=1}^{r} d_i +\omega(2g-2)r)\\
   %   &+l(\sum_{i={1+\delta_{1,g}}} ^{n} d_i + n\omega(2g-2) )
%\end{align}
%\end{equation}
%\hline 

\end{attempt}

\begin{attempt}


Fix $g \ge 2$ (but do not let $d=0$ yet). Without loss of generality let $S=\{1,2,\dots, r\}$ and $|S|=r$. We have $\delta_{1,g}=0$ and

\begin{equation} 
    x_S + \frac{ l(d+1-g - x_{[n]}  ) } {2g-2} = k
\end{equation}

Multiplying by the denominator we have,

$$(2g-2)x_S +l(d+1-g-x_{[n]}) =(2g-2)k.$$

Replace $x_i$ by terms after the action of $\lambda$, then

$$ (2g-2)[ \sum_{i=1}^{r} (d_i + \omega (2g-2) + (-1)^t x_i ] + l [d+1-g - \sum_{i=1}^{n} (d_i +\omega(2g-2)+(-1)^t x_i ] = (2g-2)k$$

multiplying by $2g-2$

$$ (2g-2)[ \sum_{i=1}^{r} (d_i + \omega (2g-2)] +(2g-2)[(-1)^{t}\sum_{i=1}^{r}( x_i )] + l [d+1-g -  (-1)^t\sum_{i=1}^{n} (x_i ) ]+l [-\sum_{i=1}^{n} (d_i +\omega(2g-2)] = (2g-2)k.$$

Rearranging,

$$(2g-2)[(-1)^{t}\sum_{i=1}^{r} x_i ] + l [d+1-g -  (-1)^t\sum_{i=1}^{n} x_i ] =(2g-2)k -
(2g-2)[ \sum_{i=1}^{r} (d_i + \omega (2g-2)]
+l [\sum_{i=1}^{n} (d_i +\omega(2g-2)]$$

Fix $t=0$, then the LHS is

$$(2g-2)[\sum_{i=1}^{r} x_i ] + l [d+1-g - \sum_{i=1}^{n} x_i ],$$

and of the form of equation (\ref{hypereq}). We want to show that the RHS is an integer after division by $2g-2$. In the definition of $P_{g,n}$ have $\omega(2g-2) + \sum_{i=1}^{n} d_i=0$.


\begin{titlemize}{Question}
 \item Can you use the definition of $P_{g,n}$ when $g=1$?
\end{titlemize}

Hence,

$$LHS=(2g-2)[k- \sum_{i=1}^{r} (d_i + \omega (2g-2)]
+l [(n-1)\omega(2g-2)]$$

So the RHS is an integer after divison by $2g-2$ and so lives on $W_D$ for a different $k$ value. 

\begin{titlemize}{Question}
 \item Questions can we determine which $S^{'}$ the $\lambda$ action moves $S$ to?
 \item what about $l$. 
 \item What happens when $t=1$?
 
 %$$LHS= (2g-2)[-\sum_{i=1}^{r} x_i ] + l [d+1-g +\sum_{i=1}^{n} x_i ] $$
 
\end{titlemize}



\end{attempt}



\printindex

\bibliographystyle{alpha}
\bibliography{bibtex}





\end{document}
